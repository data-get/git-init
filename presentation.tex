\documentclass[xcolor=svgnames,17pt]{beamer}

\usepackage[export]{adjustbox}
\usepackage{bashful}
\usepackage{bookmark}
\usepackage{colortbl} \arrayrulecolor[gray]{0.7}
\usepackage{microtype}
\usepackage{pgfpages}
\usepackage{rotating}
\usepackage{textcomp}
\usepackage{tabularx}
\usepackage{xspace}
\usepackage{verbatim}

\usepackage{fontspec}

\hypersetup{pdfpagemode=,colorlinks=true,urlcolor=blue}

%\urlstyle{same}

\newcommand*{\sizefont}[1]{%
    \ifcase#1\relax
    \or \tiny
    \or \scriptsize
    \or \footnotesize
    \or \small
    \or \normalsize
    \or \large
    \or \Large
    \or \LARGE
    \or \huge
    \or \Huge
    \fi}

%%

\newcommand*{\mybullet}{\tikz[baseline=-.6ex]\node[%
    draw,circle,inner sep = -0.15ex,fill]{.};\xspace}

%\setbeamertemplate{footline}{
%    \usebeamercolor[fg]{page number in head/foot}%
%    \usebeamerfont{page number in head/foot}%
%    \hspace*{1ex}\insertframenumber\,/\,\inserttotalframenumber\hfill
%    github.com/andrewdotn/...\ }

\newcommand*{\plainfooter}{%
    \setbeamertemplate{footline}{
        \usebeamercolor[fg]{page number in head/foot}%
        \usebeamerfont{page number in head/foot}%
        \hspace*{1ex}\insertframenumber\,/\,\inserttotalframenumber\vskip2pt}}

\makeatletter
\def\alphslide{\@alph{\intcalcAdd{1}{\intcalcSub{\thepage}{\beamer@framestartpage}}}}
\newcommand*{\plainstepfooter}{
    \setbeamertemplate{footline}{
        \usebeamercolor[fg]{page number in head/foot}%
        \usebeamerfont{page number in head/foot}%
        \hspace*{1ex}\insertframenumber\alphslide\,/\,\inserttotalframenumber\vskip2pt}}
\makeatother

\setbeamertemplate{note page}{
    \sizefont{3}
    \setlength{\parskip}{10pt}
    \insertnote
    \par}

\setbeamertemplate{navigation symbols}{}
\setbeamerfont{title}{size=\LARGE}
\setbeamerfont{frametitle}{size=\LARGE}
\setbeamerfont{framesubtitle}{size=\normalsize}

\newcommand*{\tocsection}[1]{\pdfbookmark[2]{#1}{#1}}

\lstdefinestyle{bashfulStdout}{
    basicstyle=\ttfamily,
    keywords={},
    showstringspaces=false
}%

%%

\title{git init}

\author{\texorpdfstring{%
    Andrew Neitsch}{Andrew Neitsch}}

\date{\small 2018-01-24}

\begin{document}

\tocsection{Title page}

\sizefont{4}

\begin{frame}[plain]
\titlepage
\end{frame}

\begin{frame}{Outline}
\tableofcontents
\end{frame}

\section{Version control}

\def\fillinblank{\_\_\_\_}

\begin{frame}{Basics of version control}
\begin{itemize}
\item Tracking who made what changes, when, and why, gets very complicated
\end{itemize}
\end{frame}

% \begin{frame}
% \tableofcontents[currentsection]
% \end{frame}

\begin{frame}[fragile]
\sizefont{2}
\bash[stdout,script,prefix=$\space]
git --version
\END
\bash[stdout]
git --version
\END
\end{frame}

\section{Version control with git}

\section{Digging deeper}

\begin{frame}
\begin{itemize}
\item \texttt{git cat-file -p \textit{thing}}
\item \texttt{\detokenize{git cat-file -p HEAD^{tree}}}
\item \texttt{git cat-file tree \textit{hash} | hexdump -C}
\end{itemize}

% https://git-scm.com/book/en/v2/Git-Internals-Git-Objects

\end{frame}

\section{Conclusion}

\begin{frame}{Conclusion}
\end{frame}

\section{Exercises}

\begin{frame}{Exercises}
\begin{itemize}
\item Install git
\item Create a new repo locally and commit some changes
\item Check out \href{https://github.com/py-yyc/git-init}{github.com/py-yyc/git-init}
\item Send a pull request with your change
\end{itemize}
Advanced:
\begin{itemize}
\item Extend the hash-computing code to work with directories and multiple
commits
\end{itemize}
\end{frame}

\end{document}
